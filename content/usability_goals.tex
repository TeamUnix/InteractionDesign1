\section{Usability Goals}\textit{Dennis.}\\
The below sections will be given points on a scale from 1 to 10, depending on
the importance (were 10 is very important and 1 is less important)
\begin{itemize}
	\item 10 safe to use (safety)
	\item 8 easy to learn (learnability) 
	\item 8 easy to remember how to use (memorability)
	\item 6 effective to use (effectiveness) 
	\item 5 efficient to use (efficiency/performance)
	\item 3 have good utility (utility)	
\end{itemize}
\subsection{Safety}
Most important is the safety factor as the user of the system (Jan) should be able to show the system for high-school students and others without some fiddle-fingers is hurt. Therefore everything dangerous such as: high voltage connections and sharp edges should be held inside the module boxes.
Non-technically experienced persons should also be able to connect devices to the system, not thinking about safety risks. 
\subsection{Learnability}
As the system is meant to be operated by a non-technically person, learning the system should be fast and operating the system should therefore be very intuitive. 
Also visitors should be able to do simple tasks on the system, such as turn a device on or off.
\subsection{Memorability}
Mainly the system is supposed to run by it self without any interaction. But when the system is going to be operated on or the system is showed to some visitors, the instructed person(s) should be able to do so without any preparation time.
\subsection{Effectiveness}
The effectiveness of the system is not the most important factor, but still quiet important as we are talking about a green system, which should be
affordable for the user to implement. When investing in a green system, normally it will have some information about estimated lifetime and buy time (the time
it takes the system to `buy home itself'). If the buy time is to long, maybe longer than the lifetime, the price for having such a system will be way to high. 
Visitors should also be able to see the advantage in buying a green system (environmental and money friendly), so it might affect them to be more environmental friendly. 
\subsection{Efficiency/Performance}
The performance in the system is not critical when considering it from the users perspective. It doesn't matter if it takes a few seconds before the system responds the user. But for safety reasons it should be operate very fast between the modules so no parts is harmed due to too long response time.
\subsection{Utility}
The possibility of changed a lot of parameters in the system should be kept as low as possible to easy the interaction with the system. If the user has too many ways of setting up the system it can easily confuse more than necessary. Therefor only the most important things that the user should be able to change is implemented, rest of the settings should be placed as hidden utilities. 