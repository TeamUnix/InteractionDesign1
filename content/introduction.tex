\chapter{Introduction}
\section{Interaction Design - Paulo}
Interaction design is the process of understanding and satisfying needs and desires of a person or group that will use a product being developed (can be called a goal-oriented design). When interaction sessions are held with the user throughout the whole design process, the user are able to give feedback on an early stage in the development process, or actively assist in the designing process (Participatory Design).
When creating a goal-oriented design for a certain user or a target group, goals shall be set up to understand and clarify the needs of the user or target group (Usability Goals). These goals will also give a basis for testing the final product, to validate if the user or target groups goals have been fulfilled.
\\[0.2cm]
In this project, different ways of developing a user interface, including design, prototyping and testing have been used.
A web interface and a physical interface were developed, using different methods of design and user sessions. A participatory design method for the physical interface and a prototype of the web interface was developed, with focus on the user needs.
\\[0.2cm]
This report will describe the design process in the development of the Energy Hub prototype as an academic project. The experience gained in the development of a prototype, meetings with the users and the use of different methods to design the interfaces will be described. A short description of the Energy Hub project will give an overall understanding and expected achievements.
\\[0.5cm]
Recorded sessions can be found at: \textit{http://e10.ede.hih.au.dk/index.php/Videos}
%%%%%%%%%%%%%%%%%%%%%%%%%%%%%%%
\section{Problem Statement - Theis}
The PRO3 project is divided into sub projects that each team is working on. Our project is the energy hub, the problem statement for this device is as follows (taken from the teachers wiki).
\begin{itemize}
	\item\textit{The system controller monitors all connected devices.}
	\item\textit{A connected device can either consume or produce power.}
	\item\textit{All devices must have a standardized interface, both HW/SW/Mech.}
	\item\textit{The controller must decide what happens with energy flows to and from all connected devices, in an optimum manner.}
	\item\textit{The hub must be able to recognize all connected devices.}
\end{itemize}
A new request was to make some interaction with the hub through a web page, as well on the energy hub. The user interface for this purpose is to be developed with the methods provided in the EIDE1 course.


%%%%%%%%%%%%%%%%%%%%%%%%%%%%%%%
\section{Task Description - Theis}
The hub is the central part of the overall system, all the modules is connected to this unit. The hub has to react when different events occurs in the overall system. This could be when a device starts producing energy, then the hub have to find out what to do with the extra energy. What if a consuming module starts consuming energy, or starts consuming more energy in short time. 
\\[0.2cm]
Our task is to develop a box that controls all the modules and react on the above mentioned kind of events. Because of the fact, that all modules are connected to the hub, then most of the interaction in this system is machine to machine communication. The human to machine interaction in the system is only taking place on the energy hub and on an energy web site created. The hub interaction is only for the owner, where the web site is for everyone to use, which gives some additional requirements for the user interface. Is shall be possible to see the status of the hub, the modules and to see how mush energy is being produced and consumed on the website.

%%%%%%%%%%%%%%%%%%%%%%%%%%%%%%%
\section{Stakeholder Analysis - Paulo}

Stakeholder Analysis is an important tool for developers, as the different involvement of all person on the project and on the final product is clarified. This is done by identifying persons or groups which are relevant and their level of influence on the project.
\\[0.2cm]
\textbf{Project coordinators:}\\ Morten Opprud\\ Klaus Kolle\\
\\
\textbf{Customers/Users:}\\
Jan Nielsen - Customer ( Primary User )\\
High school students ( Secondary Users )\\
Rene A. S. Josefsen ( Web interface customer )\\
\\
\textbf{Suppliers:}\\
Jens Mortensen\\
Per Lysgaard\\
\\
\textbf{Theory Advisors:}\\
Henning Slavensky\\
Ulrich Bjerre\\
Kristian Lomholdt\\
Per Lysgaard\\

\begin{figure}[h!]
 \begin{center}
  \begin{tabular}{| l | l | l |}
   \hline
    & \textbf{Has decision power} & \textbf{Has no decision power} \\ \hline
    \multirow{3}{*}{\textbf{Directly involved stakeholder}} 
    	& Klaus Kolle & Rene A. S. Josefsen\\ 
    	& Morten Opprud &  \\ 
    	& Jan Nielsen &  \\ \hline
    \multirow{2}{*}{\textbf{Not directly involved stakeholders}} 
    	&  & Jens Mortensen\\
    	& Per Lysgaard & High School Students \\ \hline
   \end{tabular}
  \end{center}
 \caption{Stakeholder Analysis table}
\end{figure}

Klaus Kolle and Morten Opprud, are the project coordinators/managers, they
have decision power over the final product and are directly involved on the
development.\\

Jan Nielsen is the customer, the primary user of the system so he has the
decision power over the final product and is enrol in all the development and design process.\\

High School Students have no decision power over the final interface since they
are the secondary users of the system. They will be one of the final users to test the system.\\

Rene A. S. Josefsen is our web-interface customer, has no decision power in the
overall system, but is direct involve in the design of the system interface.\\

Jens Mortensen is the component supplier, is not directly involved in the
project and has no decision power over the final product.\\

Per Lysgaard is not directly involved in the development of the system, but has the final decision of the budget for the system.\\

\textbf{{\Huge Who should be made happy and who should be informed?}}

%%%%%%%%%%%%%%%%%%%%%%%%%%%%%%%