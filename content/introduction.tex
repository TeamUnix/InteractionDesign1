\chapter{Introduction - Paulo}
\textbf{What can be read in this report ??? AND WHY IS IT IMPORTANT WITH INTERACTION DESIGN}
\\ \textbf{BOTH DESIGNING WEB AND PHYSICAL MODULE (Remove this after implemented!)}
\\This report will describe the design process of the development of the Energy Hub prototype as an academic project. The experience gained on the development of a prototype, meetings with the users, and the use of different methods of design will be described. A short description of the Energy Hub project will give an overall understanding and expected achievements.
\\
In the start of each section it is informed who has written it.
\\
The recorded sessions can be found here: \textit{http://e10.ede.hih.au.dk/index.php/Videos}
%%%%%%%%%%%%%%%%%%%%%%%%%%%%%%%
\section{Problem Statement - Theis}
\textbf{Insert link to wiki + explain what is says there in short therms.}


%%%%%%%%%%%%%%%%%%%%%%%%%%%%%%%
\section{Task Description - Theis}
\textbf{GOOD CONTENT, BUT NEEDS SOME REWRITING}
As developing the hub as the central part of the overall system, all the modules is connected to the hub. The hub then has to react when different events occurs in the system. That could be when a device starts producing energy, then the hub have to find out what to do with the extra energy. What if a consuming module starts consuming energy, or starts consuming more energy in short time. Our task is to develop a box that controls all the modules and react on that kind of events. Because of the fact that all modules is connected to the hub, then most of the interaction in this system is machine to machine communication. The human to machine interaction in the system is, on the hub and on a web page. The hub interaction is only for the owner, where the web page is for every one, which gives some additional requirements for the user interface. Is should be possible to see the status of the hub, the modules and to see how mush energy is producing and consuming.

%%%%%%%%%%%%%%%%%%%%%%%%%%%%%%%
\section{Stakeholder Analysis - Paulo}
\textbf{WHY DO WE MAKE THIS ANALYSIS. Maybe to know who to make happy and who to ask about what}
This section describes the influence that each individual have on the project.\\[0.2cm]
\textbf{Project coordinators:}\\ Morten Opprud\\ Klaus Kolle\\
\\
\textbf{Customers/Users:}\\
Jan Nielsen - Customer ( Primary User )\\
High school students ( Secondary Users )\\
Rene A. S. Josefsen ( Web interface customer )\\
\\
\textbf{Suppliers:}\\
Jens Mortensen\\
Per Lysgaard\\
\\
\textbf{Theory Advisors:}\\
Henning Slavensky\\
Ulrich Bjerre\\
Kristian Lomholdt\\

\begin{figure}[h!]
 \begin{center}
  \begin{tabular}{| l | l | l |}
   \hline
    & \textbf{Has decision power} & \textbf{Has no decision power} \\ \hline
    \multirow{3}{*}{\textbf{Directly involved stakeholder}} 
    	& Klaus Kolle & Rene A. S. Josefsen\\ 
    	& Morten Opprud &  \\ 
    	& Jan Nielsen &  \\ \hline
    \multirow{2}{*}{\textbf{Not directly involved stakeholders}} 
    	& High School Students & Jens Mortensen\\
    	& Per Lysgaard & \\ \hline
   \end{tabular}
  \end{center}
 \caption{Stakeholder Analysis table}
\end{figure}

Klaus Kolle and Morten Opprud, are the project coordinators/managers, they
have decision power over the final product and are directly involved on the
development.\\
\newline
Jan Nielsen is the customer, the primary user of the system so he has the
decision power over the final product and is enrroll in all the development
process.\\
\newline
High School Students have decision power over the final interface since they
are the secondary users of the system. They will be our testers.\\
\newline
Rene A. S. Josefsen is our web-interface customer, has no decision power in the
overall system, but is direct involve in the development of the system.\\
\newline
Jens Mortensen is the component supplier, is not directly involved in the
project and has no decision power over the final product.\\
\newline
Per Lysgaard is the money supplier.\\
%%%%%%%%%%%%%%%%%%%%%%%%%%%%%%%