\chapter{Process description}

This chapter describes the design process of both the physical and the web interface. A brief description of all the steps taken from scratch until a final prototype for the user.

\section{Physical Interface - Paulo}

In a user centred interaction design, the needs and desires of the user that is going to interact with the system are important information for a user focus design process.

Meeting with the user at first clarifies the project, and the developers can be introduced to the user desires and needs in a physical interface. After the meeting, the input from the user is analysed, and having the user needs in mind, a hi-fidelity physical interface is build using rapid prototype tools, such as the mBed, prototype housing and low cost components.
With the use of a participatory design, the user is able to design, with the developers help, the physical interface that meet his needs. This was a great design tool to use, since the design made by the user was completely different from the prototyped.

Participatory design don't only allow the design product to be exactly the user needs, but it can 'save' time in the product design. In this case a prototype was build that developers expected to be the user needs, but simple to interact interface was expected by the user.

Physical components such as switch, lcd, LEDs, etc. were printed and cut, this allowed the user to arranged different components on the front panel of a prototype housing. Developers were asking information from the user and helping with the design process. In the end a low-fidelity physical interface was generated and discussed with the user.

Further in this project, using the feedback from the participatory design, a hi-fidelity prototype of the physical interface is going to be build using the rapid prototype tools, so a usability test can be performed with the only user of the physical interface.

\section{Web Interface - Paulo} 

In the web interface, developers have different types of target users, and clients who desires and needs differ due to them different background and expectations. 

With a overall concept of the Energy Hub project in mind, a meeting with Rene was arranged, so his requirements could be clear. The feedback from this meeting was analysed and a web page layout was created having on focus the user requirements and the target user group. The design was inspired on Apple products and colours, they give a good user experience and easy understanding.

Since this would be a web interface used by all the teams, a preview was presented to the teams and different changes were performed on the layout design, until a final design was accepted from all teams. 

A low-fidelity prototype was made using powerpoint and linking slides. This would give the user a overview of the navigation and system information.

Two feedback session were arranged with the user and client, their feedback was used to perform changes on the layout so it could meet their expectations and needs.

In general the client ( Rene ) liked the interface layout for is easy interaction and user experience, but the data was confusing, this will be improved according to the feedback acquired.

For the user ( Jan ) the colour used were boring, and the interaction with the system wasn't intuitive. This feedback will be taken be kept in mind when improving the system, since Jan is the first user of the system.

Usability test were performed on different users with no involvement or overview of the project, this feedback is going to be a great help to improve the system, since High School students with no overview of the projects are the secondary users of the system.

Further in this project, all the feedback will be evaluated and a hi-fidelity prototype of the fully functional interface will be developed and usability tests are going to be perform on the web interface.

\section{Overall Understanding of the System - Paulo} 

To give an overall understudying of the system communication and how it will work, a drama session was performed, where students from the class were the system modules, hub and web server.

Drama session gave a overview of the system, it was useful to understand faults in the system, so them can be further corrected and the system communication improved.
\\
\\3 different scenes were created and performed:\\
- Scene 1: Hub and modules connection and startup.\\
- Scene 2: Communication Hub<->Modules, Hub<->Webserver.\\
- Scene 3: Data collision and handling.\\

In each of this scenes a real live communication and system functionalities were performed. 

Further corrections will be performed on the system according to the feedback acquired in this drama session.





