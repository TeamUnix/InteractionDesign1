\chapter{Process description}

This chapter describes the design process of both physical and web interfaces. A brief description of all the steps taken from scratch until a final prototype for the user.

\section{Physical Interface - Paulo}

A meeting was arranged with so the needs and desires of the user about the Energy Hub could be clear to us, developers. After analysing all the input from the user a first physical hi-fidelity interface was design so we would have a proposal layout for the user. 

Using a participatory design we allowed the user to design, with our help, the physical interface he would need. This was a great tool of design to use, since the design made by the user was completely different of what we thought and prototyped. 

At the beginning we let the user design without any knowledge about our prototype, and in the end of the meeting our prototype was show to the user.

Participatory design can be a great tool to meet the customer desire, since the user is designing is own product.

Further more in this project a prototype of the physical interface is going to be created using the mBed ( Rapid prototype board ), so a usability test can be arranged with the user.

\section{Web Interface - Paulo} 

After getting some ideas about what should be shown to the users and how the hub should work, a meeting was arranged with our web interface user. He gave us some requirements about what should be included in the web page.

In team we brainstorm and made a preview of how the interface should look like, this was then discussed with all the teams in the group meetings. .The designed followed was inspired in Apple products and colours since they give a good user experience and easy understanding.

We get to a final preview of the web interface, and a low-fidelity prototype was made using powerpoint and linking slides. 

A meeting was arranged with the user and is feedback was taken in consideration to make minor changes in the interface.

In general the user liked our interface for is easy interaction, understanding and user experience.

Further more a hi-fidelity prototype of the fully functional interface will be created and usability tests are going to be perform on the interface, not just the navigation it self but its functionalities.

\section{Overall Understanding of the System - Paulo} 

To give an overall understudying of the system communication and how it will work, a drama session was arranged, where students from the class were the system modules, hub and web server.

This gives a view of what happens in the system and it was helpful to understand some faults in the system, so this can be further correct.
\\
\\3 different scenes were created and performed:\\
- Scene 1: Hub and modules connection and startup.\\
- Scene 2: Communication Hub<->Modules, Hub<->Webserver.\\
- Scene 3: Data collision and handling.\\

In this drama section the system was shown and some feedback was gives by the invited teachers, this help us to understand what should be improved.