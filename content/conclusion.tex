\chapter{Conclusion}
\textbf{Written by All}\\

Before this introduction course to Interaction Design, we were a little sceptic about it, we though it would be time consuming, without giving any results from it, that could help the development of our system.

By doing the exercises proposed in the course, we realised the great tool that interaction design could be, we start seeing faults and misunderstandings between what we though was the user needs and what was really needed.

This was very useful for our project in this early phase.

Interaction design is a great tool in product development. Different methods of design were used giving different perspectives how design process works.

Participatory design is most used when developing a interface for only one user, since this user desires and toughs will be complete different from other users. This method decreases the time of design, since the developers don't have to develop several prototypes until it meets the user needs. Instead the user is a co-designer, is integrated as a developer and asked to design him self the layout/interface that he think he/she may need.

Usability tests are the most useful and powerful tool of interaction design. A must in all the projects that involves an user or a group. This gives a great feedback on what is not that intuitive as expected. Different types of user knowledges have to be selected to have the most reliable feedback, keeping always in mind the target group or user that is going to interact with the system.