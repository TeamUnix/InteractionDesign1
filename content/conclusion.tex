\chapter{Conclusion}
\textbf{Written by All}\\
The findings indicate that a participatory design is most useful when developing a interface for only one user, since this user desires and toughs will be complete different from other users. This method decreases the time of design, since the developers don't have to develop several prototypes until it meets the user needs. Instead the user is a co-designer, is integrated as a developer and asked to design him self the layout/interface that he/she think he/she may need. The findings also shows that a usability tests are the most useful and powerful tool of interaction design. A must in all the projects that involves an user or a group. This gives a great feedback on what is not that intuitive as expected. Different types of user knowledges have to be selected to have the most reliable feedback, keeping always in mind the target group or user that is going to interact with the system.
\\\\
Before this introduction course to Interaction Design, we were a little sceptic about it. We though it would be very time consuming as we could easily make an interface without help from the users and sessions with the users would just be waste of time and slow down the development of our system.\\
By doing the exercises proposed in the course, we realized what a great tool that interaction design could be, as faults and misunderstandings between what we though was the user needs and what really was his need.\\
This was very useful for our project in this early phase. At first we developed a high fidelity prototype of the hubs physical interface and though this would be easy. Just showing Jan the prototype and get a thumbs up. This was however not the case. As explained earlier, a participatory design session where scheduled. Here we learned that it can be very hard to define the needs of a user and that user sessions are very useful to define the exact user needs.
\\\\
The entire process has been very useful and a great success, since we have been able to develop a static web interface for the hub, which our tests has shown is not that hard to use. We also got feedback on small things to change, we also obtain a good picture of the physical interface for the hub, the way Jan would like it.

%Interaction design is a great tool in product development. Different methods of design were used giving different perspectives how design process works.