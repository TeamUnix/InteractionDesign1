\chapter{Conclusion}
\textbf{Written by All}
\p A participatory design is most useful when developing a interface for only one user, since this user desires and toughs can be completely different from others. This method decreases the time of design, since the developers do not have to develop several prototypes until it meets the user needs. Instead the user is a co-designer, integrated as a developer and asked to design him self the layout/interface that is going to be implemented. 
\p Usability tests are a useful and powerful tool in interaction design. A must in all projects that involves an user or an user-group. This gives a great feedback on what is not intuitive as expected. Different types of user knowledges have to be selected to have the most reliable feedback, keeping always in mind the target group or user that is going to interact with the system.
\p Before this introduction course to Interaction Design, our team was a little sceptic about it. The first thought was that it would be time consuming and would slow down the development of our system, without giving any useful feedback.
\p By doing the exercises proposed in the course, interaction design became a great tool on user focus development, since it eliminated the faults and misunderstandings between what was understood as user needs and what really was needed.
\p Interaction design was very useful to the project in this early phase. At first the development of a high fidelity prototype of the hubs physical interface was created. With this high fidelity prototype, we thought it would be easier to get the acceptance from the user, however this was not the case. As explained in the report, a participatory design session was scheduled, and we experienced, that it can be very hard to define the user needs.\p The entire design process has been useful and it have been possible to develop a static web interface for the hub. Tests have shown improvements to be made and with the help of a participatory design session, a layout of the physical interface with the exact user needs was created for further development. 