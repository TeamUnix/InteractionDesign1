\chapter{User sessions}
\section{Prototypes - Theis}
\textbf{What prototypes have been made ??? PowerPoint, HTML, Hub front, photoshop?}


%%%%%%%%%%%%%%%%%%%%%%%%%%%%%%%%%%%%%%%%%
\section{Participatory Design - Dennis}
As explained earlier, the hub module is divided into two sections, the web interface and the physical interface on the hub. 
The goal of the participatory design session with the customer Jan was to clarify the level of interaction he wants on the hub module, but also where the different things should be placed. A video of the session can be found on the wiki page (find link in the introduction section). Before the session, pictures of small buttons, LED's, screens, connecters etc. were printed out to easy to process of placing components on the front panel. Jan's job was now, with some guidance instructions, to place the components he wanted on the hub and where he wanted them. The design ended up with a clean design containing: 
\begin{itemize}
	\item 1 Power button, to turn on the hub. The button should have built in LED.
	\item 1 emergency button (powers off everything).
	\item 1 on/off switch for each module (position switch).
	\item 2 indication LED for each module (green on = module is powered on. Red on = module is powered off).
	\item A 230VAC plug is placed on the front to connect: light, phone-chargers or similar. 
\end{itemize}
To get access to the front panel of the hub, a locker should be open with a key, to protect agains fiddle-fingers. The emergency button is of cause operational all the time, and no locker needs to be opened to use it.
\\When finishing his idea about the front panel of the hub, Jan was introduced to another solution, which worked as a prototype of the finish product. Jan was asked to go through some tasks:
\begin{itemize}
	\item Turn on the hub.
	\item Connect a module.
	\item Turn on the connected module.
	\item Identify an error on a module.
	\item Repair the module.
	\item Shut down the system.
\end{itemize}
The general impression from Jan was positive, but some of functions found on the interface was unnecessary as he will primarily use a PC to check the status of the different devices (on the web interface). The placement and method of connecting new devices was fine and the same with turning on and off the hub. Instead of the pushbuttons found on the prototype, as mentioned, he wants position switches. The indication of every module was fine, but unnecessary indicators should be removed. 
%%%%%%%%%%%%%%%%%%%%%%%%%%%%%%%%%%%%%%%%%
\section{First usability tests - Theis/Paulo}
\textbf{How has the setup for these sessions been, and how did they turn out. Did we learn anything???}



%%%%%%%%%%%%%%%%%%%%%%%%%%%%%%%%%%%%%%%%%
\section{Second usability tests - Theis/Paulo}
\textbf{How has the setup for these sessions been, and how did they turn out. Did we learn anything???}

